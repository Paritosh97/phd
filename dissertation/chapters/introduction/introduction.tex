\documentclass[../../main.tex]{subfiles}
\begin{document}
\chapter{Introduction}
\label{ch:introduction}

Sign language is a primary mode of communication for the Deaf and hard-of-hearing communities. It is vital for ensuring accessibility and inclusivity in various aspects of life, such as education, media, and public services. With the advancement of technology, there is an increasing need for high-quality sign language synthesis systems that can provide naturalistic and expressive animations, bridging communication gaps between signers and non-signers.

TODO

\section{Motivation}
Despite advancements, current sign language synthesis systems struggle to capture the full complexity of sign language, including hand shapes, movements, and facial expressions. The challenge lies in creating scalable models that can adapt to the diverse linguistic and cultural contexts of sign languages across the globe.

Improving sign language synthesis is crucial for enhancing accessibility and autonomy for Deaf individuals. Recent breakthroughs in machine learning, computer vision, and natural language processing present opportunities to address these challenges and develop more advanced systems that offer greater realism and adaptability.

\section{Objectives and Contributions}

The primary objectives of this thesis are:
\begin{itemize}
    \item To develop new methods for generating more natural and expressive sign language animations.
    \item To explore machine learning techniques that enhance the scalability and adaptability of sign language synthesis models.
\end{itemize}

This research is expected to make the following contributions:
\begin{itemize}
    \item Introducing novel algorithms or models for sign language synthesis.
    \item Demonstrating improvements in the realism and comprehensibility of generated sign language.
    \item Providing a framework that can be adapted to various sign languages and contexts.
\end{itemize}

The structure of this thesis is designed to guide the reader through the research in a logical and coherent manner. Following this introductory chapter, Chapter \ref{ch:background-work} provides a comprehensive review of the background work, covering existing methods in sign language synthesis, relevant machine learning techniques, and the challenges identified in the literature. The subsequent chapters, \ref{rigging_layers}, \ref{multi_track}, \ref{facial_expressions}, \ref{intermediate_blocks}, \ref{pose_matching}, focus on the key technical contributions of this research. Chapter \ref{rigging_layers} introduces the concept of rigging layers, followed by Chapter \ref{multi_track}, which discusses the multi-track approach. Chapter \ref{facial_expressions} explores the integration of facial expressions, while Chapter \ref{intermediate_blocks} delves into the development of intermediate blocks. Chapter \ref{pose_matching} addresses pose matching techniques. Finally, Chapter \ref{conclusion} concludes the thesis by discussing the implications of the findings, their potential applications in real-world scenarios such as accessibility and education, and offering a summary of the key results along with suggestions for future research.


\section{Publications}
Work presented in this thesis has been the subject of the following publications:
\begin{cvpubs}
    \fullcite{sharma:hal-03721720}
\end{cvpubs}

\begin{cvpubs}
    \fullcite{sharma:hal-04143663}
\end{cvpubs}

\begin{cvpubs}
    \fullcite{10193413}
\end{cvpubs}

\begin{cvpubs}
    \fullcite{10.1145/3606037.3606837}
\end{cvpubs}

\begin{cvpubs}
    \fullcite{Sharma2024FacialEF}
\end{cvpubs}

\begin{cvpubs}
    \fullcite{10.1145/3658852.3659080}
\end{cvpubs}
\end{document}
