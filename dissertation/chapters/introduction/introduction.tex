\documentclass[../../main.tex]
{subfiles}
\begin{document}
\chapter{Introduction}
\label{ch:introduction}

\gls{sl} is the primary mode of communication for the Deaf and hard-of-hearing communities, playing a crucial role in communication across various aspects of life, such as education, media, and public services. However, there is a notable lack of resources and tools to fully integrate \gls{sl}s into these domains, limiting their broader usage and recognition. To address this, we must invest in technologies and systems that empower the language itself, ensuring that \gls{sl} has the tools, platforms, and visibility it deserves to function on equal terms with spoken and written languages.

\gls{sl} is a rich and complex mode of communication that relies on visual-gestural modalities, utilizing movements, facial expressions, and body postures to convey meaning. These languages are not universal but vary across regions and cultures, each with its own grammar, syntax, and lexicon. As a result, creating effective \gls{sl} synthesis systems requires a deep understanding of both linguistic intricacies and the technical demands of animation.

\section{Motivation}
\label{ch:introduction:motivation}

The motivation for this research stems from the need to work on equipping \gls{sl} with more tools that can replicate the full spectrum of \gls{sl}, including the subtle nuances of gestures and facial expressions—elements integral to the grammar and meaning of the language.

The communication gap between signers and non-signers creates barriers in various aspects of daily life. This problem is made worse by the worldwide lack of \gls{sl} interpreters. According to the European Centre for Modern Languages of the Council of Europe (ECML), approximately 750,000 hard-of-hearing people use \gls{sl} as their primary language of communication in Europe~\cite{sl_interpreters}. However, there are only 6,600 active \gls{sl} interpreters~\cite{sl_interpreters}. In nations with inadequate funding for Deaf education and interpreter training, this ratio is even worse~\cite{hrw2019}. Additionally, communication barriers during emergency situations such as the recent COVID-19 pandemic have further underscored the need for more \gls{sl} tools and resources~\cite{panko2021deaf}. Development of avatar-based automated synthesis systems can help to not only bridge this communication gap but also equip \gls{sl} with newer digital tools~\cite{kipp2011sign}.

Apart from the shortage of interpreters, \gls{sl} synthesis systems could also help in popularizing the language. Similar to how the development of the printing press and typewriters made it simpler to create and distribute written languages, synthesis systems can assist in producing more \gls{sl} content that a larger audience can consume. This can increase the visibility of \gls{sl} in the media and popularize it within non-signers as well.

Moreover, given that \gls{sl}s are intricately linked to the culture and identity of Deaf communities and are significantly context-dependent, with variations influenced by factors such as the signer's individual style and geographical region, synthesis systems can aid in capturing these subtleties for research and academic purposes~\cite{mti7100097}.

The demand for improved tools to produce \gls{sl} content in educational institutions is another motivation behind this research. At the moment, educational institutions with limited resources find it difficult to supply Deaf pupils with instructional materials~\cite{skutnabb2013linguistic}. This is even more noticeable in areas where there are fewer \gls{sl} interpreters. Automated \gls{sl} synthesis systems can make it quicker and easier to create such educational materials.

Finally, the use of virtual environments and avatars has increased due to recent developments in computer graphics and vision. As more interactions move online, there is a growing need for avatars that can communicate in \gls{sl}. This requires advances in both animation technology and the underlying linguistic models that drive these avatars. By developing better synthesis systems, it is possible to create avatars capable of meaningful communication with Deaf users, providing a richer and more inclusive online experience as well as also enabling anonymization of the content~\cite{xia2022sign}.

\section{Objectives}
\label{ch:introduction:objectives}

This research is centered on addressing the limitations of existing \gls{sl} synthesis systems and exploring new methods to enhance the naturalness and scalability of synthesized \gls{sl} animations. Specifically, this research aims to:

\begin{itemize}
    \item \textbf{Develop new methods for generating more natural and expressive \gls{sl} animations:} This research focuses on improving the quality of \gls{sl} synthesis based on linguistic input that better captures the complexities of \gls{sl}, including gestures and facial expressions, which are crucial for conveying the meaning of a \gls{sl} utterance.
    \item \textbf{Explore the scalability of \gls{sl} synthesis:} This research also addresses the labor-intensive task of creating \gls{sl} animations. A scalable synthesis system addresses this by providing a framework for creating and combining different motions to create a bigger \gls{sl} utterance.
\end{itemize}

\section{Significance of the Problem}
\label{ch:introduction:significance}

The significance of this research lies in the fact that the field of \gls{sl} synthesis is still in its early stages, with many challenges yet to be addressed. There is a potential to close the technical gap between \gls{sl} linguists and character animators, facilitating a more seamless collaboration between the two. Up to this day, \gls{sl} linguists and animators have worked in parallel but largely disconnected domains. \gls{sl} linguists focus on the linguistic structure and cultural context of \gls{sl}, while character animators concentrate on creating visually convincing movements in digital avatars. However, the complexity and nuance of \gls{sl} require a deeper integration of these two areas to produce animations that are both linguistically accurate and visually expressive.

This research addresses the synthesis problem by using linguistic descriptions as the basis for the \gls{sl} synthesis process. This formalism provides a structured approach to encoding the various components of \gls{sl}, including gestures and facial expressions, within a unified system. This research bridges the gap between the abstract linguistic representations of signs and the concrete visual expressions needed for character animation. A formal system also allows for lesser reliance on handcrafted animations and enables better scalability for synthesizing different \gls{sl} utterances.

In summary, the research can contribute to both the field of \gls{sl} linguistics and the domain of computer animation by providing a framework that unites the strengths of both disciplines. It addresses the need for more effective tools that can translate linguistic insights into \gls{sl} animations on an avatar, thus promoting \gls{sl}.

\section{Structure of the Thesis}
\label{ch:introduction:structure}

Following this introductory chapter, Chapter~\ref{ch:background_work} provides a comprehensive review of the background work, covering existing methods in describing and synthesizing \gls{sl}s, relevant techniques, and the challenges identified in the domain. This chapter sets the stage for the technical contributions that follow, offering a detailed overview of the current state of the field and the gaps that this research aims to address.

Chapter~\ref{ch:avatar_creation_pose_synthesis} introduces our pipeline for creating a signing avatar as well as generating simple animations from a linguistic description. This chapter delves into the technical aspects of creating and managing the skeletal structure of avatars, creating posture constraints from a linguistic description, as well as synthesizing on the avatar using a posture constraint optimization algorithm.

Chapter~\ref{ch:multi_track} focuses on synthesis by discussing the multi-track approach to \gls{sl} synthesis. This approach allows for the simultaneous representation of multiple aspects of a sign, such as hand movements, facial expressions, and body postures. By integrating these elements into a cohesive system, the multi-track approach enables a more scalable and natural model for \gls{sl} synthesis.

Chapter~\ref{ch:intermediate_blocks_pose_correction} addresses both the creation of intermediate blocks, template matching, and the application of deep learning techniques for pose correction. The chapter highlights the importance of intermediate blocks in ensuring fluid transitions between different animation elements, discussing how motion curves and templates contribute to achieving lifelike movements. It also introduces new techniques for generating these blocks to enhance the overall realism of the animation. Additionally, the chapter covers pose correction methods using an \gls{sl} pose prior trained on a \gls{mocap} dataset. This ensures that the synthesized animations don't break the generated postures. By presenting new approaches for pose correction, this research improves the quality of poses generated from a small amount of \gls{mocap} data. These combined techniques offer a more polished and natural representation of \gls{sl} in animation.

In Chapter~\ref{ch:facial_expressions}, the focus shifts to the synthesis of facial expressions, which are a critical component of \gls{sl}. Facial expressions convey a wealth of information in \gls{sl}, from grammatical markers to emotional cues. This chapter explores the challenges of synthesizing realistic facial expressions and presents new methods for integrating them into \gls{sl} animations.

Finally, Chapter~\ref{ch:conclusion} concludes the thesis by discussing the implications of the findings, their potential applications in real-world scenarios, and offering a summary of the key results along with suggestions for future research. This chapter also reflects on the broader impact of the research, considering how the advancements in \gls{sl} synthesis can contribute to the ongoing efforts to promote the language and inclusivity for Deaf individuals.

\end{document}
