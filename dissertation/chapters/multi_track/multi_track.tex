\documentclass[../../main.tex]{subfiles}
\begin{document}

\chapter{Multi-Track and Non-Linear Synthesis}
\label{ch:multi_track}

Human movement is a symphony of coordinated actions, where each part of the body plays a distinct role, yet all work together in harmony to create a seamless expression of intent. The essence of this coordination lies in the ability to perform multiple actions simultaneously or in carefully timed sequences, with each movement contributing to the larger tapestry of physical expression. Traditional approaches to understanding or replicating this complexity often fall short, reducing human motion to a linear series of bone rotations that lack the depth and nuance of true human movement.

The concept of multi-track control offers a different perspective—one that acknowledges the layered nature of our physical existence. It recognizes that our movements are not isolated but interwoven, with each gesture or posture influencing and being influenced by others. By allowing multiple threads of action to unfold concurrently or in precise intervals, multi-track control mirrors the natural rhythm and flow of life. It captures the interplay between stillness and motion, the balance of tension and release, and the silent dialogue between different parts of the body as they move together yet independently. In doing so, it brings us closer to understanding the profound complexity of human movement, honoring the subtlety and grace inherent in every action.

Similarly, non-linearity in movement reflects the dynamic and adaptive nature of human actions. Just as life rarely follows a straightforward path, our movements often involve layers of simultaneous actions that interact in complex ways. In fields like video editing or film production, non-linear control allows creators to work with multiple layers of footage, sounds, and effects, enabling them to craft scenes where various elements unfold together, contributing to a richer and more cohesive narrative.

In the realm of sign language, the non-linear and multi-track nature of human movement is particularly evident. Sign language involves a combination of simultaneous gestures, facial expressions, and body movements, all working together to convey meaning. By adopting a non-linear, multi-track approach, we can more accurately capture the intricate coordination involved in signing, where gestures and expressions occur in parallel, influencing and overlapping with each other. This approach respects the complexity and interconnectivity inherent in sign language, allowing for a more nuanced and realistic portrayal of its dynamic and expressive nature. In doing so, it brings us closer to truly understanding and representing the fluidity and richness of sign language as it is naturally used, honoring its depth and complexity as a form of human communication.

This chapter aims to present a novel approach to sign language synthesis using multi-track and non-linear editing techniques. It showcases improvements in animation quality and flexibility, demonstrating how these techniques preserve the dynamics of individual animation blocks and enhance procedural generation capabilities.

The content of this chapter is structured as follows: We introduce the key concepts of multi-track timeline control and non-linear editing in sign language synthesis. The \emph{Multi-Track Timeline} section covers related work, AZee score synchronization, and conflict resolution techniques. The \emph{Non-Linear Editing} section discusses its role in enhancing animation flexibility, along with rules for block ordering. In \emph{Implementation and Results}, we describe the integration within Blender’s editor and evaluate AZee synthesis against benchmarks. Finally, the \emph{Conclusion and Future Prospects} section summarizes the advantages of our approach and suggests directions for future research.

\section{Multi-Track Timeline}

todo

\subsection{Related Work}

todo

\subsubsection{Existing low-level AZee synthesizor}

todo fabrizio, cant leverage shortcuts, etc etc

\subsubsection{Paula}

todo john, cant leverage low level, etc etc

\subsubsection{Outside Sign Language Synthesis}

todo Matthis

\subsubsection{Non-Linear Editing}
todo Non-linear editing refers to the ability to access and edit any frame of a video clip or animation independently of the sequence. This flexibility allows for more sophisticated and precise adjustments, essential for achieving high-quality animations.

\subsection{AZee Synced Score to Multi-Track Timeline}

todo

\subsection{AZee and Non-Linear Editing}

todo

\subsection{Resolving Block Conflicts}

todo


\subsubsection{Pre-animated blocks}
todo

\subsection{Block Ordering}

Pre-animated blocks put first

\subsubsection{Rule 1: Timely Evaluation}
\textbf{Problem:} Overlapping blocks with different start times.\\
\textbf{Solution:} Evaluate blocks chronologically to maintain logical sequence.

\subsubsection{Rule 2: Constraint Precedence}
\textbf{Problem:} Overlapping blocks starting simultaneously.\\
\textbf{Solution:} Give precedence to placement constraints over orientation constraints.

\subsubsection{Rule 3: Second Pass for Transpaths}
\textbf{Problem:} Blocks with transpath constraints.\\
\textbf{Solution:} Evaluate these blocks in a second pass after other constraints are resolved.

\subsubsection{Rule 4: Second Pass for Holds}
\textbf{Problem:} Blocks with hold constraints.\\
\textbf{Solution:} Evaluate these blocks in a second pass to ensure dependent constraints are maintained.

Cases where blocks do not overlap or affect different bone chains can be evaluated independently. This includes non-overlapping blocks and constraints such as morph and look, which act independently from other constraints.

\section{Implementation and Results}

The multi-track timeline is implementated in blender's non-linear editor. Each AZee Score, when animated, generates a blender action which is put on the non-linear editor as a block with duration specified by the AZee \emph{sync rules}

\subsection{Evaluation}

we dont evaluate sign language synthesis, we evaluate azee synthesis

against mediapipe 

against paula

against rosetta mocap

against state of the art and sgnify

Examples of synthesized AZee descriptions, comparing flattened and non-flattened synthesis. Highlight the preservation of dynamics and any potential issues encountered. Discuss the use of Frobenius distance as a metric for evaluating the accuracy of the synthesized animations.

\section{Conclusion and Future Prospects}

\subsection{Conclusion}
Summarize the benefits of the proposed multi-track synthesis algorithm, emphasizing the preservation of dynamics and improved naturalness in sign language animations. Highlight the contributions of this approach to the field of sign language synthesis.

\subsection{Future Prospects}
Discuss potential extensions to this work, such as integrating top-down techniques, handling morph constraints, and reducing robotic movements through advanced techniques like ambient noise analysis and style transfer. Outline possible future research directions and improvements.

\end{document}
