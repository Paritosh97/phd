\providecommand{\main}{../..}
\documentclass[../../main]{subfiles}
\begin{document}
\chapter{Facial Expressions}
\label{ch:facial_expressions}

Facial expressions are a crucial element of communication in both spoken and \gls{sl}s. In \gls{sl}s, they serve a dual purpose: conveying the emotional state of the signer and encoding grammatical information such as negation, and emphasis. While spoken languages rely on tone and intonation for these functions, \gls{sl}s depend heavily on \gls{non_manual_signals}, particularly facial expressions, to fully convey the meaning of an utterance. Therefore, creating accurate and expressive facial animations in signing avatars is essential for producing realistic and comprehensible \gls{sl} content.

However, synthesizing facial expressions for signing avatars presents significant challenges due to the complexity and subtlety of these expressions. Unlike manual signs, which involve distinct hand and arm movements, facial expressions require the coordinated movement of numerous facial muscles, each contributing to the overall expression. Different facial features, such as eyebrows, eyes, and mouth, often move independently yet in harmony to produce coherent expressions.

Previous chapters focused on the synthesis of skeletal features in \gls{sl}. In contrast, this chapter centers on the synthesis of facial expressions from the AZee model, emphasizing their critical role in the conveyance of grammatical information in \gls{sl}s.

The chapter is structured as follows: todo

\section{Adding Facial Expressions to AZee}
\label{ch:facial_expressions:adding_facial_expressions_to_azee}

The AZee model does have a specification for \emph{morph} constraints. However, a proper specification for facial expressions was missing. Thus, our first taks was to extend the set of \emph{morphs} in the AZee. We used the 40 brèves corpus~\cite{challant2024extending, challant2022first} as a reference for our study. The 40 brèves corpus consists of 23 facial expressions in the form of AZee rules (figure~\ref{fig:all_expressions_breves}). The frequency of each expression in the corpus is shown in figure~\ref{fig:expression_frequency}.

\begin{figure}
    \centering
    \includegraphics[width=0.8\textwidth]{chapters/facial_expressions/images/all_expressions_breves.png}
    \caption{All facial expressions in the 40 brèves corpus}
    \label{fig:all_expressions_breves}
\end{figure}

\begin{figure}
    \centering
    \includegraphics[width=0.8\textwidth]{chapters/facial_expressions/images/expression_frequency.png}
    \caption{Frequency of facial expressions in the 40 brèves corpus}
    \label{fig:expression_frequency}
\end{figure}

\subsection{Defining AZee Rules in terms of Action Units}
\label{ch:facial_expressions:defining_azee_rules_in_terms_of_action_units}

We discussed the \gls{facs} earlier in section~\ref{ch:background_work:sign_language_synthesis:3d_techniques:avatar_animation:face_animation:emotion_recognition}. \gls{facs} is a comprehensive system for describing facial expressions in terms of action units (AUs), which represent the activation of specific facial muscles. Each AU corresponds to a specific movement~\annex{facs_aus}. Thus, our first goal is to break down each AZee rule into its constituent AUs. 

We use the AU detector by~\cite{luo2022learning} on mediapipe extractions of still images for each of the facial expressions from the corpus (figure~\ref{ch:facial_expressions:fig:face_detect}).

%todo more detailed diagram(images -> mediapipe -> AU detector)
\begin{figure}
    \centering
    \includegraphics[width=0.8\textwidth]{chapters/facial_expressions/images/face_detect.png}
    \caption{Facial action unit detection using mediapipe and the AU detector by~\cite{luo2022learning} for \emph{big-threatening}}
    \label{ch:facial_expressions:fig:face_detect}
\end{figure}

While it is a good starting point, manual adjustments were required by a linguist to ensure accuracy. This can be seen in figure~\ref{cfig:face_adjust} for the expression \emph{big-threatening}.

% todo face adjust diagram(example medaipe AU face retargeted on a template mesh, then adjusted by a linguist)
\begin{figure}
    \centering
    \includegraphics[width=0.8\textwidth]{chapters/facial_expressions/images/face_adjust.png}
    \caption{Manual adjustment of facial expressions to ensure accuracy.}
    \label{ch:facial_expressions:fig:face_adjust}
\end{figure}

\section{Blendshape Creation}
\label{ch:facial_expressions:blendshape_creation}

Once the AUs were identified, we used FACSHuman~\cite{gilbert2021facshuman} blendshapes as reference to create shapekeys in blender that correspond to each AU (figure~\ref{ch:facial_expressions:fig:facshuman_blendshapes}). Shape keys (figure~\ref{ch:facial_expressions:fig:shape_keys}) are essentially different versions of a 3D model, each representing a specific change in vertices. By blending and interpolating these keys together, we can create a wide range of mesh shapes.

Since the FACSHuman blendshape work on any MakeHuman template mesh, this technique of synthesis is avatar independant. However, since the original model doesn't cover all the blendshapes(for all the face meshes), some blendshapes were created manually. The complete process involved iterating between manual adjustments and automated tools to ensure that the blendshapes accurately captured the intended expressions. The creation of blendshapes also involved ensuring that they could be seamlessly combined to create complex expressions. For example, the blendshape for AU4 (Brow Lowerer) was designed to work in conjunction with AU6 (Cheek Raiser) and AU12 (Lip Corner Puller) to create expressions of anger or determination. This required careful coordination of the vertex movements across different blendshapes to avoid unnatural deformations or artifacts in the final animation.

\begin{figure}
    \centering
    \includegraphics[width=0.8\textwidth]{chapters/facial_expressions/images/facshuman_blendshapes.png}
    \caption{FACSHuman blendshapes used as reference for creating shapekeys in blender.}
    \label{ch:facial_expressions:fig:facshuman_blendshapes}
\end{figure}

\begin{figure}
    \centering
    \includegraphics[width=0.8\textwidth]{chapters/facial_expressions/images/shape_keys.png}
    \caption{Blender interface. (a) Shape Key properties (b) 3D Viewport (c) Non-linear Editor (d) Action Editor (e) AZee editor}
    \label{ch:facial_expressions:fig:shape_keys}
\end{figure}

Lastly, we extended the low-level AZee language morph set with 94 blendshapes corresponding to most of the action units in the FACS system (figure~\ref{tab:added_units}) along with some additional blendshapes for the tongue. A few action units were not added~\ref{tab:not_added} since they were controlled by other skeletal constraints. Some morphs are also alternative blendshapes for the same action unit~\ref{tab:alternative_blendshapes} but with a different effect on the face.

\begin{table}[h]
    \centering
    \scriptsize
    \begin{tabular}{|c|l|c|l|}
        \hline
        \textbf{AU Code} & \textbf{Description}                     & \textbf{AU Code} & \textbf{Description}                      \\ \hline
        AU1              & Inner Brow Raise                        & AU1\_L            & Inner Brow Raise (Left)                   \\ \hline
        AU1\_R           & Inner Brow Raise (Right)                & AU2              & Outer Brow Raise                          \\ \hline
        AU2\_L           & Outer Brow Raise (Left)                 & AU2\_R            & Outer Brow Raise (Right)                  \\ \hline
        AU4              & Brow Lowerer                            & AU5              & Upper Lid Raise                           \\ \hline
        AU5\_L           & Upper Lid Raise (Left)                  & AU5\_R            & Upper Lid Raise (Right)                   \\ \hline
        AU6              & Cheek Raise                             & AU6\_L            & Cheek Raise (Left)                        \\ \hline
        AU6\_R           & Cheek Raise (Right)                     & AU7              & Lids Tight                                \\ \hline
        AU7\_L           & Lids Tight (Left)                       & AU7\_R            & Lids Tight (Right)                        \\ \hline
        AU8              & Lips Toward Each Other                  & AU9              & Nose Wrinkle                              \\ \hline
        AU9\_L           & Nose Wrinkle (Left)                     & AU9\_R            & Nose Wrinkle (Right)                      \\ \hline
        AU10             & Upper Lip Raiser                        & AU10\_L           & Upper Lip Raiser (Left)                   \\ \hline
        AU10\_R          & Upper Lip Raiser (Right)                & AU11             & Nasolabial Furrow Deepener                \\ \hline
        AU11\_L          & Nasolabial Furrow Deepener (Left)       & AU11\_R           & Nasolabial Furrow Deepener (Right)        \\ \hline
        AU12             & Lip Corner Puller                       & AU12\_L           & Lip Corner Puller (Left)                  \\ \hline
        AU12\_R          & Lip Corner Puller (Right)               & AU13             & Sharp Lip Puller                          \\ \hline
        AU13\_L          & Sharp Lip Puller (Left)                 & AU13\_R           & Sharp Lip Puller (Right)                  \\ \hline
        AU14             & Dimpler                                 & AU14\_L           & Dimpler (Left)                            \\ \hline
        AU14\_R          & Dimpler (Right)                         & AU15             & Lip Corner Depressor                      \\ \hline
        AU15\_L          & Lip Corner Depressor (Left)             & AU15\_R           & Lip Corner Depressor (Right)              \\ \hline
        AU16             & Lower Lip Depress                       & AU17             & Chin Raiser                               \\ \hline
        AU18             & Lip Pucker                              & AU19             & Tongue Show                               \\ \hline
        AU20             & Lip Stretch                             & AU20\_L           & Lip Stretch (Left)                        \\ \hline
        AU20\_R          & Lip Stretch (Right)                     & AU21             & Neck Tightener                            \\ \hline
        AU22\_25\_up\_down & Lip Funneler (Both Lips) & AU23             & Lip Tightener                             \\ \hline
        AU24             & Lip Presser                             & AU25             & Lips Part                                 \\ \hline
        AU25\_L          & Lips Part (Left)                        & AU25\_R           & Lips Part (Right)                         \\ \hline
        AU26             & Jaw Drop                                & AU27             & Mouth Stretch                             \\ \hline
        AU28             & Lips Suck                               & AU29             & Jaw Thrust                                \\ \hline
        AU30\_L          & Jaw Sideways (Left)                     & AU30\_R           & Jaw Sideways (Right)                      \\ \hline
        AU31             & Jaw Clencher                            & AU32             & Bite                                      \\ \hline
        AU33             & Blow                                    & AU34             & Puff                                      \\ \hline
        AU35             & Cheek Suck                              & AU36             & Tongue Bulge                              \\ \hline
        AU37             & Lip Wipe                                & AU38             & Nostril Dilate                            \\ \hline
        AU39             & Nostril Compress                        & AU43             & Eye Closure                               \\ \hline
        AU43\_L          & Eye Close (Left)                        & AU43\_R           & Eye Close (Right)                         \\ \hline
    \end{tabular}
    \caption{Added blend shapes} 
    \label{tab:added_units} 
\end{table}

\begin{table}
    \centering
    \scriptsize
    \begin{tabular}{|c|l|c|l|}
        \hline
        \textbf{AU Code} & \textbf{Description}                     & \textbf{AU Code} & \textbf{Description}                      \\ \hline
        AU40             & Sniff                                   & AU41             & Lid Droop                                 \\ \hline
        AU42             & Slit                                    & AU44             & Squint                                    \\ \hline
        AU45             & Blink                                   & AU46             & Wink                                      \\ \hline
        AU51             & Head Turn Left (IK controlled)          & AU52             & Head Turn Right (IK controlled)           \\ \hline
        AU53             & Head Up (IK controlled)                 & AU54             & Head Down (IK controlled)                 \\ \hline
        AU55             & Head Tilt Left (IK controlled)          & AU56             & Head Tilt Right (IK controlled)           \\ \hline
        AU57             & Head Forward (IK controlled)            & AU58             & Head Back (IK controlled)                 \\ \hline
        AU61             & Eyes Turn Left (lookat constraint)      & AU62             & Eyes Turn Right (lookat constraint)       \\ \hline
        AU63             & Eyes Up (lookat constraint)             & AU64             & Eyes Down (lookat constraint)             \\ \hline
    \end{tabular}
    \caption{Action units skipped} 
    \label{tab:not_added} 
\end{table}


\vspace{1cm}

\begin{table}
    \centering
    \scriptsize
    \begin{tabular}{|c|l|c|l|}
        \hline
        \textbf{AU Code} & \textbf{Description}                     & \textbf{AU Code} & \textbf{Description}                      \\ \hline
        AU1\_a            & Inner Brow Raise (Alternative)          & AU2\_a            & Outer Brow Raise (Alternative)           \\ \hline
        AU4\_a            & Brow Lowerer (Alternative A)            & AU4\_b            & Brow Lowerer (Alternative B)             \\ \hline
        AU6\_a            & Cheek Raise (Alternative A)             & AU6\_b            & Cheek Raise (Alternative B)              \\ \hline
        AU9\_a            & Nose Wrinkle (Alternative A)            & AU12\_a           & Lip Corner Puller (Alternative A)        \\ \hline
        AU12\_b           & Lip Corner Puller (Alternative B)       & AU17\_a           & Chin Raiser (Alternative A)              \\ \hline
        AU18\_a           & Lip Pucker (Alternative)                & AU25\_a           & Lips Part (Alternative A)                \\ \hline
        AU22\_25\_upper  & Lip Funneler (Upper Lip)     & AU22\_25\_down   & Lip Funneler (Bottom Lip)  \\ \hline
        AU26\_lip\_down   & Jaw Drop Bottom Lip Down (Alternative)  & AU26\_tongue\_down & Jaw Drop Tongue Down (Alternative)       \\ \hline
        AU26\_tongue\_out & Jaw Drop Tongue Out (Alternative)       & AU26\_a           & Jaw Drop (Alternative)                   \\ \hline
        AU28\_a           & Lips Suck (Upper Lip)                   & AU28\_bottom      & Lips Suck (Lower Lip)                    \\ \hline
        tongue\_back\_up  & Tongue Back Up                          & tongue\_out       & Tongue Out                               \\ \hline
        tongue\_up        & Tongue Up                               & tongue\_wide      & Tongue Wide                              \\ \hline
    \end{tabular}
    \caption{Alternative blend shapes} 
    \label{tab:alternative_blendshapes}
\end{table}



\subsection{Motion Curves for Blendshapes}
\label{ch:facial_expressions:motion_curves_for_blendshapes}

After creating the blendshapes, we generated motion curves (more discussed in chapter~\ref{ch:intermediate_blocks_pose_correction}) to control how these shapes are animated over time. Motion curves define the changes in the blendshape's influence over the course of an animation, allowing for smooth transitions between different facial expressions.

We extended our intermediate block generation algorithm to include motion curves for facial morphs. This involved creating additional curves that specify the timing and intensity of facial movements based on the template. By controlling the acceleration and deceleration of these movements, we were able to create more naturalistic animations that reflect the dynamic nature of facial expressions.

For example, in the expression "big-threatening," the motion curves were designed to gradually increase the influence of the blendshapes corresponding to AU10 (Upper Lip Raiser) and AU25 (Lips Part) while simultaneously decreasing the influence of AU4 (Brow Lowerer) as the expression transitions from a neutral state to one of aggression (figure~\ref{ch:facial_expressions:fig:motion_curve_example}). This careful modulation of the blendshape influences over time resulted in an expression that not only looked realistic but also conveyed the intended emotional and grammatical cues effectively.

\begin{figure}
    \centering
    \includegraphics[width=0.8\textwidth]{chapters/facial_expressions/images/motion_curve_example.png}
    \caption{Motion curves controlling the blendshape influences for the expression "big-threatening."}
    \label{ch:facial_expressions:fig:motion_curve_example}
\end{figure}

%todo metahuman, flame, etc.

\section{Results and Evaluation}
\label{ch:facial_expressions:results}

Synthesis of all the facial expressions from the 40 brèves corpus can be seen in table~\ref{tab:facial_expressions}. The expressions were created by combining the relevant blendshapes to produce realistic and expressive facial animations.

% todo finish table
\begin{table}
    \centering
    \begin{tabular}{|c|c|c|}
        \hline
        \textbf{AZee rule} & \textbf{Description} & \textbf{Expression} \\
        \hline
        todo & todo & todo \\
        todo & todo & todo \\
        \hline
    \end{tabular}
    \caption{AZee rules and their synthesized facial expressions}
    \label{tab:facial_expressions}
\end{table}

\section{Evaluation}
\label{ch:facial_expressions:evaluation}

Table~\ref{tab:facial_expressions_evaluation} shows the subjective evaluation by a linguist of the facial expressions synthesized using the AZee framework. The expressions were assessed based on their accuracy, expressiveness, and effectiveness in conveying the intended emotional and grammatical cues.

\begin{table*}
    \centering
    \begin{tabular}{|l|p{8cm}|}
    \hline
    \textbf{Expression} & \textbf{Limitations} \\
    \hline
    \texttt{almost-reaching} & Mouth modeling unconvincing. \\
    \hline
    \texttt{continuously} & "Pffff" air and cheek puff difficult, neutral eyebrows. \\
    \hline
    \texttt{do-you-realise} & Thick eyebrow issue. \\
    \hline
    \texttt{it-is-a-shame} & Mouth expression not quite real. \\
    \hline
    \texttt{most-probably} & Less visible teeth preferred, thick eyebrow issue. \\
    \hline
    \texttt{much-almost-too-much} & Frowning eyebrows and lack of eye wrinkles not convincing. \\
    \hline
    \texttt{nothing-sticks-out} & Tucked lips difficult to model. \\
    \hline
   \texttt{something-sticks-out} & Interpreted as confusion, mouth modeling limitation. \\
    \hline
    \texttt{trouble-disturbance} & Frowning eyebrows difficult, mouth "rising" hard to model, result not convincing. \\
    \hline
    \texttt{uneasy-awkward} & Tongue tip out with slightly open mouth hard to model, unconvincing. \\
    \hline
    \texttt{with-chaos} & Single cheek blow/puff and alternating eye blinks hard without animation. \\
    \hline
    \texttt{with-no-precision} & Upper lip over lower and mouth near nose unmodellable. \\
    \hline
    \texttt{with-surprise} & Cannot lower lower eyelid fully, thick eyebrow issue. \\
    \hline
    \texttt{with-uncertainty} & Appears sadder than uncertain, thick eyebrow issue. \\
    \hline
    \texttt{with-worry} & Lack of wrinkles around nose/forehead. \\
    \hline
    \end{tabular}
    \caption{Limitations for the synthesized blendshapes}
    \label{tab:facial_expressions_evaluation}
\end{table*}

We observe that expressions that involved subtle mouth movements, such as "it-is-a-shame" or "something-sticks-out," were more challenging to model accurately, highlighting areas for further refinement.

Table~\ref{tab:facial_expressions_quantitative} shows the quantitative evaluation of the facial expressions using the Frechet Expression Distance (FED). We also fitted FLAME~\cite{FLAME} on the ground truth facial expressions~\ref{ch:facial_expressions:fig:spectre} and compared them to the expressions synthesized using our method framework. The FED scores provide a measure of the similarity between the ground truth and synthesized expressions, with lower scores indicating a closer match.

\begin{figure}
    \centering
    \includegraphics[width=0.8\textwidth]{chapters/facial_expressions/images/sgnify.png}
    \caption{Facial expressions generated using FLAME by fitting SPECTRE}
    \label{ch:facial_expressions:fig:spectre}
\end{figure}

\begin{table}
    \centering
    \begin{tabular}{|c|c|}
        \hline
        \textbf{Expression} & \textbf{FED Score} \\
        \hline
        big-threatening & todo \\
        \hline
    \end{tabular}
    \caption{Quantitative evaluation of facial expressions synthesized}
    \label{tab:facial_expressions_quantitative}
\end{table}

The FED scores indicate that the synthesized expressions closely matched the ground truth expressions, demonstrating the effectiveness of our method in capturing the nuances of facial movements.

We also compare the synthesized utterance (with facial expressions) with the original utterance (without facial expressions) to evaluate the impact of facial expressions on \gls{sl} comprehension. For this we synthesize the following AZee description.

\begin{verbatim}    
    todo
\end{verbatim}

The following link shows the comparison between the two versions (and the SGNify version for inference) of the utterance: \href{todo}. We observe that the added facial expressions todo...

\section{Conclusion}
\label{ch:facial_expressions:conclusion}

Facial expressions play a crucial role in \gls{sl} communication, conveying both emotional and grammatical information. In this chapter, we presented a method for synthesizing facial expressions using the AZee framework, focusing on the creation of blendshapes from action units (AUs) and the generation of motion curves to control these shapes over time. Our method involved analyzing AUs from the FACS system, creating blendshapes based on these AUs, and generating motion curves to animate the blendshapes. The synthesized facial expressions were evaluated subjectively by a linguist and quantitatively using the Frechet Expression Distance (FED) metric, demonstrating the effectiveness of our method in capturing the nuances of facial movements. The results show that the synthesized expressions closely matched the ground truth expressions, highlighting the potential of our method for enhancing the realism and expressiveness of signing avatars. Future work will focus on refining the blendshapes and motion curves to improve the accuracy and naturalness of the facial expressions, as well as exploring the integration of emotion recognition systems to generate facial expressions that align with the emotional tone of the signed message. By continuing to develop and refine our method, we aim to create more realistic and effective facial animations for \gls{sl} synthesis, enhancing the overall quality and accessibility of \gls{sl} content.


\end{document}