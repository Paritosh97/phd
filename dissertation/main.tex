\documentclass[english,12pt,a4paper]{book}
\usepackage[utf8]{inputenc}
\usepackage{fontenc}
\usepackage[english]{babel}
\usepackage[default,oldstyle, scale=.95]{opensans} % Using Open Sans font
\usepackage{amsmath}
\usepackage{amsfonts}
\usepackage{newunicodechar}
\usepackage{listings}
\usepackage[acronym,section]{glossaries}
\usepackage[automake, acronym]{glossaries-extra}
\usepackage{fancyhdr}
\usepackage{svg}
\usepackage{amssymb}
\usepackage{natbib}
\usepackage[hyphens]{url}
\usepackage{rotating} 
\usepackage{tablefootnote}
\usepackage{longtable}
\usepackage[pdf]{graphviz}
\usepackage{fp}
\usepackage{xcolor}
\usepackage{forest}
\definecolor{Prune}{RGB}{99,0,60}
\definecolor{B1}{RGB}{49,62,72} 
\definecolor{C1}{RGB}{124,135,143}
\definecolor{D1}{RGB}{213,218,223}
\definecolor{A2}{RGB}{198,11,70}
\definecolor{B2}{RGB}{237,20,91}
\definecolor{C2}{RGB}{238,52,35}
\definecolor{D2}{RGB}{243,115,32}
\definecolor{A3}{RGB}{124,42,144}
\definecolor{B3}{RGB}{125,106,175}
\definecolor{C3}{RGB}{198,103,29}
\definecolor{D3}{RGB}{254,188,24}
\definecolor{A4}{RGB}{0,78,125}
\definecolor{B4}{RGB}{14,135,201}
\definecolor{C4}{RGB}{0,148,181}
\definecolor{D4}{RGB}{70,195,210}
\definecolor{A5}{RGB}{0,128,122}
\definecolor{B5}{RGB}{64,183,105}
\definecolor{C5}{RGB}{140,198,62}
\definecolor{D5}{RGB}{213,223,61}
\usepackage{pgfplots}
\usepackage{alphabeta}
\usepackage{mdframed}
\usepackage{multirow} 
\usepackage{pdfpages}
\usepackage{lscape}
\usepackage{multicol} 
\usepackage{tikz}
\usetikzlibrary{shapes.geometric, arrows}
\usepackage{graphicx}
\usepackage[absolute]{textpos} 
\usepackage{colortbl}
\usepackage{array}
\usepackage{geometry}
\usepackage{titlesec}
\usepackage{hyperref}
\usepackage{textgreek}
\usepackage{algorithm}
\usepackage{algpseudocode}
\usepackage{subcaption}
\usepackage{subfiles}
\setlength{\emergencystretch}{3em}
\hypersetup{
    colorlinks=true,
    linkcolor=black,
    urlcolor=purple
}

\pagestyle{plain}

\pgfplotsset{compat=1.17}

\makeglossaries

\begin{document}
\providecommand{\main}{.}

\begin{titlepage}

%\thispagestyle{empty}

\newgeometry{left=6cm,bottom=2cm, top=1cm, right=1cm}

\tikz[remember picture,overlay] \node[opacity=1,inner sep=0pt] at (-13mm,-135mm){\includegraphics{Frame-ups.pdf}};

%*****************************************************
%******** NUMÉRO D'ORDRE DE LA THÈSE À COMPLÉTER *****
%******** POUR LE SECOND DÉPOT                   *****
%*****************************************************

\color{white}

\begin{picture}(0,0)
\put(-152,-743){\rotatebox{90}{\Large \textsc{THESE DE DOCTORAT}}} \\
\put(-120,-743){\rotatebox{90}{NNT : 2020UPASA001}}
\end{picture}
 
%*****************************************************
%**  LOGO  ÉTABLISSEMENT PARTENAIRE SI COTUTELLE
%**  CHANGER L'IMAGE PAR DÉFAUT **
%*****************************************************
\vspace{-14mm} % à ajuster en fonction de la hauteur du logo
%\flushright \includegraphics[scale=1]{logo2.png}

%*****************************************************
%******************** TITRE **************************
%*****************************************************

\flushright
\vspace{10mm} % à régler éventuellement
\color{Prune}
\fontfamily{cmss}\fontseries{m}\fontsize{22}{26}\selectfont
  \Huge 

\normalsize
\color{black}
~\\
\Large{Sign Language synthesis by a decreasing granularity system from AZee} \\
\small{\textit{Synthèse de langue des signes par un système d'animation à granularité décroissante à partir d'AZee}} \\
%*****************************************************

\fontfamily{fvs}\fontseries{m}\fontsize{8}{12}\selectfont

\vspace{1.5cm}

\normalsize
\textbf{Thèse de doctorat de l'université Paris-Saclay} \\

\vspace{6mm}

\small École doctorale n°580 : sciences et technologies de l’information et de la communication (STIC)\\
\small Spécialité de doctorat: informatique\\
\small Graduate School : Informatique et sciences du numérique, Référent : Faculté des sciences d’Orsay \\
\vspace{6mm}

\footnotesize Thèse préparée dans l'unité(s) de recherche \textbf{LISN} (Université Paris-Saclay, CNRS), sous la direction de \textbf{Michael FILHOL}, Chargé de recherche CNRS \\
\vspace{15mm}

%\textbf{Thèse soutenue à Paris-Saclay, le 5 décembre 2024, par}\\
\bigskip
\Large {\color{Prune} \textbf{Paritosh SHARMA}} % Changer le Prénom et le NOM

%************************************
\vspace{\fill} % ALIGNER LE TABLEAU EN BAS DE PAGE
%************************************

\bigskip

\flushleft
\small {\color{Prune} \textbf{Composition du jury}}\\
{\color{Prune} \scriptsize {Membres du jury avec voix délibérative}} \\
\vspace{2mm}
\scriptsize
\begin{tabular}{|p{7cm}l}
\arrayrulecolor{Prune}
\textbf{Nicolas SABOURET} & Président \& Examinateur\\ 
Professeur, Université Paris-Saclay \\
\textbf{Floris Roelofsen} &  Rapporteur \& Examinateur \\ 
Professor, University of Amsterdam \\ 
\textbf{Ludovic Hoyet} &  Rapporteur \& Examinateur \\ 
Chargé de recherche, INRIA \& Université de Rennes \\ 
\textbf{Catherine Pelachaud} &  Examinatrice \\ 
Directrice de recherche, CNRS - ISIR \& Sorbonne University \\ 
\textbf{Hui-Yin Wu} &  Examinatrice \\ 
Chargé de recherche, Inria \& Université Côte d'Azur \\ 
\textbf{Fabrizio Nunnari} &  Examinateur \\ 
Senior Researcher, German Research Center for Artificial Intelligence (DFKI) \\ 
 

\end{tabular} 

\end{titlepage}


% page des résumés à garder en 2ème page. Si les résumés sont trop longs pour tenir sur une seule et même page, on peut mettre un résumé par page
\thispagestyle{empty}
\newgeometry{top=1.5cm, bottom=1.25cm, left=2cm, right=2cm}
\fontfamily{rm}\selectfont

\lhead{}
\rhead{}
\rfoot{}
\cfoot{}
\lfoot{}

\noindent 
%*****************************************************
%***** LOGO DE L'ED À CHANGER IMPÉRATIVEMENT *********
%*****************************************************
\includegraphics[height=2.45cm]{logo_ups_STIC.png}
\vspace{1cm}
%*****************************************************
\fontfamily{cmss}\fontseries{m}\selectfont

\small

\begin{mdframed}[linecolor=Prune,linewidth=1]

\textbf{Titre:} Synthèse de langue des signes par un système d'animation à granularité décroissante à partir d'AZee

\noindent \textbf{Mots clés:} Animation Graphique, Signeur Virtuel, Langue des Signes

\vspace{-.5cm}
\begin{multicols}{2}
\noindent \textbf{Résumé:}

Les récents progrès dans la synthèse de la langue des signes ont permis de se rapprocher d'une génération plus naturelle et expressive, bien que des défis importants subsistent. Les approches actuelles se concentrent principalement sur une synthèse au niveau lexical, ce qui peut conduire à des productions manquant de fluidité et de continuité. Un défi majeur dans la synthèse de la langue des signes réside dans la capacité à capter la diversité des granularités présentes dans le discours, allant des gestes globaux aux mouvements plus subtils, qui véhiculent des nuances sémantiques spécifiques. Pour répondre à cette problématique, nous proposons un nouveau système de granularité décroissante fondé sur le modèle AZee, qui structure la langue des signes selon des niveaux hiérarchiques de granularité, du plus grossier au plus fin.

Dans le cadre de cette thèse, nous examinons l'application de ce système de granularité décroissante à la synthèse de la langue des signes. L'objectif principal de notre travail est de développer un système autonome capable de générer des séquences en langue des signes qui s'adaptent à divers contextes et exigences des utilisateurs. Contrairement aux approches classiques qui se reposent sur des gestes statiques et prédéfinis, notre méthode ajuste dynamiquement le niveau de granularité au cours de la synthèse, permettant ainsi de produire une langue des signes à la fois expressive et adaptée au contexte discursif.

Notre recherche vise à approfondir la compréhension et l'utilisation du formalisme AZee dans le cadre de la synthèse de la langue des signes. Nous évaluons l'efficacité de notre approche en la comparant aux synthétiseurs précédemment développés sur la base d'AZee. Les résultats de nos travaux montrent qu’un système de granularité décroissante constitue une solution évolutive prometteuse pour la synthèse de la langue des signes, ouvrant ainsi des perspectives intéressantes pour des recherches futures dans ce domaine.
\end{multicols}

\end{mdframed}

\newpage

\vspace{8mm}

\begin{mdframed}[linecolor=Prune,linewidth=1]

\textbf{Title:} Sign Language Synthesis by a Decreasing Granularity System from AZee

\noindent \textbf{Keywords:} Computer animation, Signing Avatars, Sign Language

\begin{multicols}{2}
\noindent \textbf{Abstract:}

Recent developments in Sign Language synthesis have brought the field closer to generating natural and expressive sign language, yet challenges remain. Current models often focus on word-level synthesis, resulting in output that can lack fluidity and naturalness. A major obstacle in Sign Language synthesis is the need to capture the varying granularities of a discourse, ranging from broad gestures to subtle movements that convey specific meanings. To overcome this, we propose a novel decreasing granularity system based on the AZee model, which organizes Sign Language into hierarchical levels of granularity, from coarse to fine.

In this thesis, we investigate the application of this decreasing granularity system to Sign Language synthesis. Our primary objective is to develop an autonomous system capable of generating Sign Language sequences that are adaptable to different contexts and user needs. Unlike conventional approaches that rely on static, predefined gestures, our method dynamically adjusts the level of granularity during synthesis, ensuring that the generated Sign Language is both contextually appropriate and expressive.

Throughout our research, we focus on deepening the understanding and utility of the AZee formalism in Sign Language synthesis. We evaluate the effectiveness of our approach by comparing it with previous AZee based synthesizer. Our findings demonstrate that a decreasing granularity system does offer a promising scalable solution to synthesize Sign Language, presenting a strong avenue for future research in this area.

\end{multicols}
\end{mdframed}

% Chapter titles
\titleformat{\chapter}[display]
{\normalfont\huge\bfseries\color{black}}
{\chaptertitlename\ \thechapter}
{20pt}{\Huge}
[\vspace{2ex}]

% Section titles
\titleformat{\section}[hang]{\bfseries\Large}{\thesection}{0.5em}
{\vspace{0.1ex}}

% Subsection titles
\titleformat{\subsection}[hang]{\bfseries\large}{\thesubsection}{0.75em}
{\vspace{0.1ex}}

% Subsubsection titles
\titleformat{\subsubsection}[hang]{\bfseries\normalsize}{\thesubsubsection}{0.75em}
{\vspace{0.1ex}}

% Paragraph titles
\titleformat{\paragraph}[runin]{\bfseries\small}{\theparagraph}{1em}{}

% Subparagraph titles
\titleformat{\subparagraph}[runin]{\bfseries\footnotesize}{\thesubparagraph}{1em}{}

\newpage
\thispagestyle{empty}
\vspace*{\fill}
\begin{center}
    \textit{"The only true wisdom is in knowing you know nothing."}\\
    --- \textbf{Socrates}
\end{center}
\vspace*{\fill}
\newpage

\subfile{acknowledgements/acknowledgements.tex}

\subfile{resume/resume.tex}

\newgeometry{top=4cm, bottom=4cm, left=2cm, right=2cm}

\newpage
\tableofcontents
\newpage
\listoffigures
\newpage
\listoftables
\newpage

% Include glossary and acronyms after the index pages
\newacronym{sl}{SL}{Sign Language}
\newacronym{mocap}{MoCap}{Motion Capture}
\newacronym{nlp}{NLP}{Natural Language Processing}
\newacronym{asl}{ASL}{American Sign Language}
\newacronym{lsf}{LSF}{Langue des Signes Française (French Sign Language)}
\newacronym{iswa}{ISWA}{International SignWriting Alphabet }
\newacronym{hamnosys}{HamNoSys}{Hamburg Sign Language Notation System}
\newacronym{nle}{NLE}{Non-Linear Editior}
\newacronym{stmc}{STMC}{Spatio-Temporal Motion Collage}
\newacronym{fk}{FK}{Forward Kinematics}
\newacronym{ik}{IK}{Inverse Kinematics}
\newacronym{ccd}{CCD}{Cyclic Coordinate Descent}
\newacronym{fabrik}{FABRIK}{Forward And Backward Reaching Inverse Kinematics}
\newacronym{pfnn}{PFNN}{Phase-Functioned Neural Networks}
\newacronym{sgplvm}{SGPLVM}{Scaled Gaussian Process Latent Variable Models}
\newacronym{vae}{VAE}{Variational Autoencoder}
\newacronym{nsvq}{NSVQ}{Noise Substitution Vector Quantization}
\newacronym{svg}{SVG}{Scalable Vector Graphics}
\newacronym{smplx}{SMPL-X}{Skinned Multi-Person Linear Model eXtended}

\newglossaryentry{utterance}{
    name=Utterance,
    description={A complete unit of signed communication in a sign language, similar to a spoken sentence in spoken language but conveyed through visual-gestural means.}
}

\newglossaryentry{azee_score}{
    name=Score,
    description={AZee's representation of a full or partial SL utterance (just like a musical score)}
}

\newglossaryentry{azee_interpreter}{
    name=AZee interpreter,
    description={A program that can interpret an AZee expression and produce an AZee score.}
}

\newglossaryentry{posture_constraint}{
    name=Posture constraint,
    description={A limitation on an avatar’s body.}
}

\newglossaryentry{cstr_score}{
    name=Constraint Score,
    description={An AZee Score which contains posture constraints.}
}

\newglossaryentry{glosses}{
    name=Glosses,
    description={A gloss in sign language represents signs using written words or symbols.}
}

\newglossaryentry{signing_space}{
    name=Signing space,
    description={The space in front of the signer where signs are produced.}
}

\newglossaryentry{site}{
    name=Site,
    description={A specific location on the avatar's mesh}
}

\newglossaryentry{latent_space}{
    name=Latent Space,
    description={Represents abstract, compressed features from input(images, motion, etc.) data}
}

\printglossary[type=\acronymtype,title={Acronyms}] % Print acronyms glossary
\printglossary[type=main, title={Glossary}] % Print main glossary
\newpage

\newgeometry{top=4cm, bottom=4cm, left=4cm, right=4cm}

\subfile{chapters/introduction/introduction.tex}
\subfile{chapters/background_work/background_work.tex}
\subfile{chapters/avatar_creation_pose_synthesis/avatar_creation_pose_synthesis.tex}
\subfile{chapters/multi_track/multi_track.tex}
\subfile{chapters/intermediate_blocks_pose_correction/intermediate_blocks_pose_correction.tex}
\subfile{chapters/facial_expressions/facial_expressions.tex}
\subfile{chapters/conclusion/conclusion.tex}

\newgeometry{top=4cm, bottom=4cm, left=4cm, right=4cm}

\subfile{annex/annex.tex}

\bibliographystyle{plain}
\bibliography{main}

\end{document}