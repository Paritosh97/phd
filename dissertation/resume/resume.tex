\documentclass[a4paper,12pt]{report}

\begin{document}

\chapter*{Résumé}

Ce travail de thèse, intitulé “Synthèse de langue des signes par un système d'animation à granularité décroissante à partir d'AZee,” explore les défis et les opportunités liés à la synthèse de la langue des signes, avec une attention particulière à la scalabilité et au naturel des animations. Sous la direction du Dr Michael FILHOL, cette recherche s’inscrit dans une perspective visant à valoriser la langue des signes en l’équipant d’outils numériques avancés et, dans un second temps, à améliorer l’accessibilité des personnes sourdes et malentendantes grâce à des outils numériques avancés. En exploitant des méthodologies innovantes et des concepts linguistiques, cette thèse propose une solution qui se positionne à l’intersection entre la informatique et la linguistique.

La problématique principale abordée dans cette thèse est la suivante : “Comment améliorer la scalabilité et le naturel dans la synthèse de la langue des signes ?” Cette question revêt une importance cruciale, car elle touche non seulement aux enjeux de communication et d’inclusion sociale, mais aussi à la nécessité de développer des technologies capables de reproduire fidèlement les nuances et subtilités des langues des signes. À cette fin, un cadre méthodologique basé sur le modèle AZee a été développé. AZee, en tant que système linguistique, offre une modélisation riche et intuitive des langues des signes, permettant de traduire des structures sémantiques complexes en animations compréhensibles et précises.

Le système proposé repose sur une approche inspirée de la granularité décroissante, permettant une spécification progressive de l'énoncé tout en offrant la possibilité d'améliorer la scalabilité et la naturalité de la synthèse. Par exemple, un signe comme "armoire" peut être représenté avec des structures plus détaillées, telles que l'orientation et la position de la main, la configuration des doigts, etc. Cette approche permet d’affiner les détails linguistiques et gestuels en fonction des besoins spécifiques tout en restant flexible pour s’adapter à différents contextes.

Les résultats obtenus démontrent le potentiel du système en termes de scalabilité et de naturalité par rapport aux systèmes d'animation procédurale existants. Ces résultats ont été validés à travers des évaluations qualitatives que j'ai réalisées, et les animations ont été jugées plus précises. Bien qu’aucune évaluation avec des personnes sourdes n'ait été effectuée, le système présente un potentiel important pour des applications pratiques dans divers domaines, tels que l’éducation inclusive, la formation professionnelle, la création de contenus accessibles et la réhabilitation pour les personnes atteintes de troubles de la communication.

Une des contributions majeures de cette thèse est le développement du module complémentaire Blender ainsi que l'intégration avec l’intégration avec l'éditeur AZVD (AZee Verbalizing Diagrams), qui permet de dessiner des discours en langue des signes. Ces outils permettront de créer rapidement du contenu en langue des signes, de configurer un avatar et de produire davantage de contenu. Parmi les autres contributions importantes figurent l'automatisation du rigging des personnages, l’animation multi-pistes, des animations plus naturelles utilisant des modèles définis et des captures de mouvement, ainsi que l'utilisation d'un VAE pour affiner les poses. Ces améliorations augmentent considérablement l'efficacité de la création de contenu tout en garantissant des résultats de haute qualité.

En conclusion, cette thèse contribue à l’état de l’art dans le domaine de la synthèse de la langue des signes, en proposant un cadre novateur qui allie rigueur linguistique, innovation technologique et adaptabilité. Les perspectives futures incluent non seulement l’évaluation continue et l’extension du système pour couvrir un éventail plus large de langues des signes, mais aussi l’exploration de nouvelles approches pour générer les animations. Cela permettra de créer des systèmes encore plus inclusifs, capables de répondre aux besoins variés d’une communauté mondiale de locuteurs en langue des signes.

\end{document}
